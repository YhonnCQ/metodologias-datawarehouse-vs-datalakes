\documentclass[twoside,twocolumn]{article}

\usepackage{blindtext} 
\usepackage{graphicx}
\usepackage[sc]{mathpazo} 
\usepackage[T1]{fontenc} 
\linespread{1.05} 
\usepackage{microtype} 


\usepackage[spanish,english]{babel} 


\usepackage[hmarginratio=1:1,top=32mm,columnsep=20pt]{geometry} 
\usepackage[hang, small,labelfont=bf,up,textfont=it,up]{caption} 
\usepackage{booktabs} 


\usepackage{lettrine} 


\usepackage{enumitem} 
\setlist[itemize]{noitemsep} 


\usepackage{abstract} 
\renewcommand{\abstractnamefont}{\normalfont\bfseries} 
\renewcommand{\abstracttextfont}{\normalfont\small\itshape} 


\usepackage{titlesec} 
\renewcommand\thesection{\Roman{section}} % 
\renewcommand\thesubsection{\roman{subsection}} 
\titleformat{\section}[block]{\large\scshape\centering}{\thesection.}{1em}{} 
\titleformat{\subsection}[block]{\large}{\thesubsection.}{1em}{} 


\usepackage{fancyhdr} 
\pagestyle{fancy} 
\fancyhead{} 
\fancyfoot{} 
\fancyhead[C]{Lorem Ipsum \today} 
\fancyfoot[RO,LE]{\thepage} 


\usepackage{titling} 

%----------------------------------------------------------------------------------------
%	TILULOS
%----------------------------------------------------------------------------------------


\setlength{\droptitle}{-4\baselineskip} 

\pretitle{\begin{center}\Huge\bfseries} 
\posttitle{\end{center}} 
\title{Comparison of Datawarehouse elaboration methodologies vs. Datalake elaboration methodologies} 
\author{
	Valdivia Guzman, Alejandra Maria\\
	\and
	Pazos Alarcón, Christian Joshua\\
	\and
	Farfan Colque, Mathius Omar\\
	\and
	Condori Quispe, Yhónn Joel\\
}
\date{\today} 
\renewcommand{\maketitlehookd}{
\selectlanguage{spanish} 
\begin{abstract}
\noindent 
Lorem ipsum dolor sit amet, consectetur adipiscing elit. Morbi vulputate tempus molestie. 
\end{abstract}
\selectlanguage{english} 
\begin{abstract}
\noindent 
Lorem ipsum dolor sit amet, consectetur adipiscing elit. Morbi vulputate tempus molestie. 
\end{abstract}
}

%----------------------------------------------------------------------------------------

\begin{document}

% Print the title
\maketitle

%----------------------------------------------------------------------------------------
%	Introduction
%----------------------------------------------------------------------------------------

\section{Introduction}

\lettrine[nindent=0em,lines=3]{L}orem ipsum dolor sit amet, consectetur adipiscing elit.
Morbi vulputate tempus molestie.

%----------------------------------------------------------------------------------------
%	State of Art
%----------------------------------------------------------------------------------------

\section{State of Art}

\subsection{Datawarehouse}

Kimball's methodology, called Dimensional Modeling, is based on what is called the Business Dimensional
Lifecycle. This methodology is considered one of the favorite techniques when building a Data Warehouse\cite{mdawa}.

In the Dimensional Model, models of tables and relationships are constituted with the purpose of optimizing
decision making, based on queries made in a relational database that are linked to the measurement or a
set of measurements of the results of the business processes\cite{mdawa}.

\subsection{Features}

This DW project life cycle is based on four basic principles:

\begin{itemize}	
	
	\item Focus on the business: Concentrate on identifying business requirements and their associated
	value, and use these efforts to develop strong relationships with the business, sharpening the business
	analysis and consultative competence of the implementers\cite{mdawa}.
	\item Build an adequate information infrastructure: Design a single, integrated, easy-to-use, high-performance
	information base that will reflect the wide range of business requirements identified in the company\cite{mdawa}.
	\item Deliver in significant increments: create the data warehouse (DW) in deliverable increments in
	6 to 12 month timeframes. Use the business value of each identified element to determine the order of
	application of the increments. In this the methodology resembles agile software construction methodologies\cite{mdawa}.
	\item Deliver the complete solution: provide all the elements necessary to deliver value to business users.
	To begin with, this means having a robust, well-designed, quality-tested, and accessible data warehouse.
	You must also provide ad hoc query tools, advanced reporting and analysis applications, training, support
	training, support, support, website and documentation\cite{mdawa}.
	
\end{itemize}

\subsection{Data Lake}

Existing reviews on data lake architectures commonly distinguish pond and zone Architectures.

\subsection{Pond Architecture}

Inmon designs a data lake as a set of data ponds\cite{Inmon}. A data pond can be viewed as a subdivision of a data lake
dealing with data of a specific type. According to Dixon’s specifications, each data pond is associated with a
specialized storage system, some specific data processing and conditioning (i.e., data transformation/preparation)
and a relevant analysis service.

\begin{center}
	\includegraphics[width=7cm]{./images/pondDL}
\end{center}

\subsection{Zone Architecture}

zone architectures assign data to a zone according to their degree of refinement\cite{Giebler}. For instance, Zaloni’s data
lake\cite{LaPlante} adopts a six-zone architecture.

\begin{center}
	\includegraphics[width=7cm]{./images/zoneDL}
\end{center}

\begin{itemize}	
	
	\item The transient loading zone deals with data under ingestion. Here, basic data quality checks are performed.
	\item The raw data zone handles data in near raw format coming from the transient zone.
	\item The trusted zone is where data are transferred once standardized and cleansed.
	\item From the trusted area, data move into the discovery sandbox where they can be accessed by data
	scientists through data wrangling or data discovery operations.
	\item On top of the discovery sandbox, the consumption zone allows business users to run “what if” scenarios
	through dashboard tools.
	\item The governance zone finally allows to manage, monitor and govern metadata, data quality, a
	data catalog and security.
	
\end{itemize}
%----------------------------------------------------------------------------------------
%	Conclusions
%----------------------------------------------------------------------------------------

\section{Conclusions}
\begin{itemize}	
	
	\item Lorem ipsum dolor sit amet, consectetur adipiscing elit. Morbi vulputate tempus molestie.
	\item Lorem ipsum dolor sit amet, consectetur adipiscing elit. Morbi vulputate tempus molestie.

\end{itemize}

%----------------------------------------------------------------------------------------
%	References
%----------------------------------------------------------------------------------------

\bibliographystyle{plain} 
\bibliography{references} 
\end{document}